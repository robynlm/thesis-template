\chapter{General Relativity} \label{sec: GR}

I put punctuation around equations because they are part of a sentence. I shall show you a beautiful equation
\begin{equation} \label{eq: EFE}
    G_{\alpha\beta} = \kappa T_{\alpha\beta},
\end{equation}
that is very beautiful. And how are each of these defined?
\begin{equation} \label{eq: Ricci id and conservation}
    2\nabla_{[\mu}\nabla_{\nu]} u^\alpha = {R^{\alpha}}_{\beta\mu\nu}u^\beta,
    \;\;\;\;\;\;\;\;\;\;\;\;
    \nabla_\alpha T^{\alpha\beta} = 0,
\end{equation}
and with faerie dust.

\section{Section} \label{sec: GR: sectionname}

\subsection{Sub Section}

\subsubsection{Sub Sub Section, the \& symbol in a title is nice}

\subsubsection{Cosmological constant in a title \texorpdfstring{$\Lambda$}{TEXT}}

To separate things more I use bullet points. Pretty symbol, pifont ding 118, to make clear that these
\begin{itemize}[label=\ding{118}]
    \item are the points I am making.
    \item With long long long long long long long long long long long long long long long long long long long text showing indentation
\end{itemize}

For a sub bullet I use the ding 70 bullet point and define the leftmargin to give a hierarchal look. I can also define itemindent such that there is no indentation should the text following the bullet be long
\begin{itemize}[label={\ding{70}}, itemindent=4em, leftmargin=0em]
    \item \textbf{First:} point, with long long long long long long long long long long long long long long long long long long long text showing indentation
    \item \textbf{Second:} point
\end{itemize}

Or if I want a more title like look I can separate them out
\begin{itemize}[label=\ding{118}, leftmargin=5em]
\item \textbf{sub sub sub section}
\end{itemize}
and put some text in between
\begin{itemize}[label=\ding{118}, leftmargin=5em]
\item \textbf{sub sub sub section}
\end{itemize}
