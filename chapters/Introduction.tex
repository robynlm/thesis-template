\chapter{Introduction}

To site someone I use either: \citep{I.Newton_1687} if I just need to include a reference or \citet{I.Newton_1687} if I mention this in the sentence. I create links like this: 
\Eqref{eq: EFE}
\Figref{fig: Szekeres delta}
\Tabref{tab: FD coef}
\Chapref{sec: GR}
\Secref{sec: GR: sectionname}
\Appref{app: analyticsolution} but if I want to reference multiple things at the same time I need to write it out in full: Eq.~(\ref{eq: EFE}, \ref{eq: Ricci id and conservation}) or Chapter~\ref{sec: GR}, \ref{sec: NR} and \ref{sec: Cosmo}
and to have footnotes\footnote{This is what I do. And I make sure these are full sentences.}.
The first time I use an accronym I give the full name: General Relativity (GR).

\section{Structure}

Here I explain what happens in the chapters: the introduction chapters are: Chapter~\ref{sec: GR}, \ref{sec: NR} and \ref{sec: Cosmo}, research chapters based on papers are Chapter~\ref{sec: EBWeyl} and \ref{sec: Sim}, and in particular I tell you exactly which parts have my original contributions.